\documentclass[12pt, letterpaper]{article}

\usepackage{amsfonts}
\usepackage{amsthm}
\usepackage{amsmath}
\usepackage{mathrsfs}
\usepackage{ amssymb }
\usepackage[margin=1in,footskip=0.25in]{geometry}
\usepackage{hyperref}
\usepackage{cleveref}
\newtheorem{thm}{Teorema}
\newtheorem{lemma}[thm]{Lema}
\newtheorem{corollary}{Corolário}
\newtheorem{remark}{Obs}
\newtheorem{proposition}{Proposição}
\newtheorem{example}{Exemplo}
\renewcommand*{\proofname}{Proof}
\newtheorem{definition}{Definição}

\title{Aula17}
\date{}
\begin{document}
\maketitle
\section*{Resumo}
    \begin{itemize}
      \item Correlação de variáveis aleatórias
      \item Propriedades da correlação
      \item Distribuição de funções de vetores aleatórios
      \item Integral com mudança de variáveis
    \end{itemize}
\section{Correlação de variáveis aleatórias}
\begin{definition}
  Sejam X,Y variáveis aleatórias com variância finita, \[\rho_{X,Y} = Cor(X,Y) = \frac{Cov(X,Y)}{DP(X)DP(Y)},\]
  onde \(Cov(X,Y) = \mathbb{E}[XY]-\mathbb{E}[X]\mathbb{E}[Y] \wedge\ DP(X) = \sqrt{Var(X)} \)
\end{definition}

\begin{proposition}[propriedades da correlação]
  \begin{enumerate}
    \item  \(-1\leq \rho_{X,Y}\leq 1 \)
    \item X,Y independentes \(\Rightarrow \rho_{X,Y} = 0\)
    \item \(\rho_{X,Y} = 1 \Rightarrow \exists\ a > 0 \wedge b \in \mathbb{R}\ |\ Y = aX + b\)
    \item \(\rho_{X,Y} = -1 \Rightarrow \exists\ a < 0 \wedge b \in \mathbb{R}\ |\ Y = aX + b\)
    \end{enumerate}

\end{proposition}


\end{document}
